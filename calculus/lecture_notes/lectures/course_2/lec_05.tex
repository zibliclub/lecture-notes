\lesson{5}{от 22 сен 2023 10:29}{Продолжение}


\begin{theorem}[О структуре касательного пространства]\label{theorem:2}
    Пусть $S$ -- $k$-мерная поверхность в $\R^n, \ x_0 \in S$. Тогда касательное пространство $TS_{x_0}$ в точке $x_0$ состоит из направляющих векторов касательных к гладким кривым на поверхности $S$, проходящих через точку $x_0$.
\end{theorem}

\begin{proof}
    Пусть $x = x(t)$ -- гладкая кривая в $\R^n$, то есть
    \[
        \left\{\begin{array}{l}
            x^1 = x^1(t) \\
            \vdots       \\
            x^n = x^n(t)
        \end{array}\right., \ t \in \R.
    \]

    $ x_0 = x(t_0) $. Касательный вектор в точке $x_0$ к кривой имеет вид:
    \[
        \left(\begin{matrix}
                \frac{dx^1}{dt}(t_0) \\
                \vdots               \\
                \frac{dx^n}{dt}(t_0)
            \end{matrix}\right) = \left(\begin{matrix}
                x^{1'}(t_0) \\
                \vdots      \\
                x^{n'}(t_0)
            \end{matrix}\right).
    \]
    \begin{enumerate}
        \item Пусть $S$ -- $k$-мерная поверхность, задана системой уравнений $F(x) = 0$ и пусть $x = x(t)$ -- гладкая кривая на $S$. Покажем, что вектор $x'(t_0) = \left(\begin{matrix}
                          \frac{dx^1}{dt}(t_0) \\
                          \vdots               \\
                          \frac{dx^n}{dt}(t_0)
                      \end{matrix}\right): \ x'(t_0) \in TS_{x_0}, \ x_0 = x(t_0)$, то есть покажем, что $x'(t_0)$ удовлетворяет уравнению $F_x'(t_0)\cdot \xi=0$.

              Так как кривая $x = x(t)$ лежит на $S$, то $F\big(x(t)\big) = 0$ -- верно. Продифференцируем $F\big(x(t)\big) = 0$ по $t$ в точке $x_0$:
              \[
                  F_x'(x_0) \cdot x'(t_0) = 0,
              \]
              это и есть уравнение касательного пространства, то есть $x'(t_0)$ удовлетворяет уравнению касательного пространства $F_x'(x_0)\cdot \xi = 0$.

        \item Пусть $\xi = (\xi^1,\xi^2,\ldots,\xi^n) \in TS_{x_0}$, то есть $\xi$ удовлетворяет уравнению $F_x'(x_0)\cdot \xi = 0$

              Покажем, что $\exists$ гладкая кривая $l$ на поверхности $S$:
              \begin{itemize}
                  \item $x_0 \in l$
                  \item $\xi$ ялвяется направляющим вектором касательной к $l$ в точке $x_0$
              \end{itemize}

              Поверхность $S$ задана системой уравнений:
              \begin{equation}\label{eq:12}
                  \left\{\begin{array}{l}
                      F^1(x) = 0 \\
                      \vdots     \\
                      F^{n-k}(x) = 0
                  \end{array}\right.
              \end{equation}

              Пусть
              \begin{equation*}
                  \left|\begin{matrix}
                      \frac{\delta F^1}{\delta x^{k+1}}     & \cdots & \frac{\delta F^1}{\delta x^n}     \\
                      \vdots                                & \ddots & \vdots                            \\
                      \frac{\delta F^{n-k}}{\delta x^{k+1}} & \cdots & \frac{\delta F^{n-k}}{\delta x^n}
                  \end{matrix}\right| (x_0) \ne 0.
              \end{equation*}

              По теореме о неявной функции, система \ref{eq:12} эквивалентна системе:
              \begin{equation}\label{eq:13}
                  \left\{\begin{array}{l}
                      x^{k+1} = f^1(x^1,\ldots,x^k) \\
                      \vdots                        \\
                      x^n = f^{n-k}(x^1,\ldots,x^k)
                  \end{array}\right.
              \end{equation}

              Обозначим $u = (x^1,\ldots,x^k), \ v = (x^{k+1},\ldots,x^n)$, тогда \ref{eq:13} имеет вид:
              \[
                  v = f(u).
              \]

              Тогда по утверждению касательное пространство задается уравнениями:
              \begin{equation}\label{eq:14}
                  \left\{\begin{array}{l}
                      x^{k+1} = x_0^{k + 1} + \frac{\delta f^1}{\delta x^1}(x_0)\cdot(x^1-x_0^1) + \ldots + \frac{\delta f^1}{\delta x^k}(x_0)\cdot(x^k - x_0^k) \\
                      \vdots                                                                                                                                     \\
                      x^n = x_0^n + \frac{\delta f^{n-k}}{\delta x^1}(x_0)\cdot(x^1 - x_0^1) + \ldots + \frac{\delta f^{n-k}}{\delta x^k}(x_0)\cdot(x^k - x_0^k)
                  \end{array}\right.
              \end{equation}

              Пусть
              \[
                  \eta = \left(\begin{matrix}
                          \eta'      \\
                          \vdots     \\
                          \eta^k     \\
                          \eta^{k+1} \\
                          \vdots     \\
                          \eta^n
                      \end{matrix}\right) = \left(\begin{matrix}
                          x^1 - x_0^1         \\
                          \vdots              \\
                          x^k - x_0^k         \\
                          x^{k+1} - x^{k+1}_0 \\
                          \vdots              \\
                          x^n - x_0^n
                      \end{matrix}\right).
              \]

              Тогда система \ref{eq:14} примет вид:
              \begin{equation}\label{eq:15}
                  \left\{\begin{array}{l}
                      \eta^{k+1} = \frac{\delta f^1}{\delta x^1}(x_0) \cdot \eta^1 + \ldots + \frac{\delta f^1}{\delta x^k}(x_0)\cdot \eta^k \\
                      \vdots                                                                                                                 \\
                      \eta^{n} = \frac{\delta f^{n-k}}{\delta x^1}(x_0) \cdot \eta^1 + \ldots + \frac{\delta f^{n-k}}{\delta x^k}(x_0)\cdot \eta^k
                  \end{array}\right.
              \end{equation}

              Таким образом, если вектор $\xi \in TS_{x_0}$, то он полностью определяется своими первыми $k$ координатами, а остальные можно волучить с помощью системы \ref{eq:15}.

              Построим кривую в $\R^n$, то есть зададим ее уравнением $x = x(t)$:
              \begin{equation}\label{eq:16}
                  l: \ \left\{\begin{array}{l}
                      x^1 = x_0^1 + \xi^1t  \\
                      \vdots                \\
                      x^k = x_0^k + \xi^k t \\
                      \left.\begin{array}{l}
                                x^{k+1} = f^1(x_0^1 + \xi^1 t, \ldots, x_0^k + \xi^k t) \\
                                \vdots                                                  \\
                                x^{n} = f^{n-k}(x_0^1 + \xi^1 t, \ldots, x_0^k + \xi^k t)
                            \end{array}\right\} \ v = f(u)
                  \end{array}\right.
              \end{equation}

              Пусть точка $x_0$ соответствует параметру $t = 0$:
              \[
                  x(0) = \left\{\begin{array}{l}
                      x^1 = x_0^1                       \\
                      \vdots                            \\
                      x^k = x_0^k                       \\
                      x^{k+1} = f^1(x_0^1,\ldots,x_0^k) \\
                      \vdots                            \\
                      x^n = f^{n-k}(x_0^1, \ldots, x_0^k)
                  \end{array}\right.,
              \]
              то есть кривая проходит через точку $x_0$.

              Далее, функция $f$ удовлетворяет условию $v = f(u) \iff F(u,v) = 0$. Тогда $F(u,f(u)) = 0\implies l$, заданная системой \ref{eq:16}, $l \subset S$.
              \[
                  (\text{\ref{eq:16}})_t': \ x_t'(0) = \left(\begin{matrix}
                          \xi^1                                                                                                  \\
                          \vdots                                                                                                 \\
                          \xi^k                                                                                                  \\
                          \frac{\delta f^1}{\delta x^1}(x_0)\cdot \xi^1 + \ldots + \frac{\delta f^1}{\delta x^k}(x_0)\cdot \xi^k \\
                          \vdots                                                                                                 \\
                          \frac{\delta f^{n-k}}{\delta x^1}(x_0)\cdot \xi^1 + \ldots + \frac{\delta f^{n-k}}{\delta x^k}(x_0)\cdot \xi^k
                      \end{matrix}\right) = \left(\begin{matrix}
                          \xi^1 \\ \vdots \\ \xi^k \\ \xi^{k+1} \\ \vdots \\ \xi^n
                      \end{matrix}\right).
              \]

              Таким образом построили гладкий путь, лежащий на поверхности $S$, проходящий через точку $x_0 \in S$, вектор $x'(t_0)$ -- его касательный вектор $\in TS_{x_0}$.
    \end{enumerate}
\end{proof}

\section{Условный экстремум функции многих переменных}

\begin{task}
    Дана функция $u = f(x^1,\ldots,x^n)$ и дана поверхность, заданная уравнениями:
    \begin{equation}\label{eq:17}
        \left\{\begin{array}{l}
            F^1(x^1,\ldots,x^n) = 0 \\
            \vdots                  \\
            F^m(x^1,\ldots,x^n) = 0
        \end{array}\right.
    \end{equation}

    Нужно найти точку $x_0 = (x_0^1,\ldots,x_0^n)$, в которой:
    \[
        f(x_0^1,\ldots,x_0^n) = \underset{(\min)}{\max}f(x^1,\ldots,x^n),
    \]
    где $\max \ (\min)$ берется по всем точкам $(x_0^1,\ldots,x_0^n)$, удовлетворяющих уравнениям \ref{eq:17}.
\end{task}

\begin{task}[Геометрическая формулировка]

    Пусть система \ref{eq:17} задает в пространстве $\R^n \ m$-мерную поверхность $S$. Найти точку $x_0 \in S: \ \exists U_x(x_0) = U(x_0)\cap S: \ \forall x \in U_s(x_0)$:
    \[
        \underset{x_0 - \max}{f(x) \leqslant f(x_0)}\text{ (или }\underset{x_0 - \min}{f(x) \geqslant f(x_0)}\text{)}
    \]
\end{task}

\begin{definition}[Линия уровня ($c$-уровень)]
    Пусть $f:D\rightarrow\R, \ D \subset \R^n$ -- область. \emph{Линией уровня ($c$-уровнем)} функции $f$ называется множество
    \[
        N_c = \big\{x\in D \ \big| \ f(x) = c\big\}.
    \]
\end{definition}

\begin{lemma}\label{lemma:1}
    Если $x_0$ -- точка условного локального экстремума для функции $f$ и $x_0$ не является критической для функции $f$ (то есть $df(x_0)\ne0$), то касательное пространство $TS_{x_0}\subset TN_{x_0}$, где
    \[
        N_{x_0} = \big\{x \in D \ \big| \ f(x) = f(x_0)\big\},
    \] -- поверхность уровня, проходящая через $x_0$.
\end{lemma}

\begin{proof}
    Пусть $\xi \in TS_{x_0}$. Пусть $x = x(t)$ -- гладкая кривая на $S: \ x(0) = x_0, \ x'(0) = \xi$ (по теореме \ref{theorem:2}).

    Так как точка $x_0$ -- условный экстремум для функции $f$, то точка $t = 0$ есть локальный экстремум для функции $f\big(x(t)\big)$ (по теореме Ферма, потом нужно добавить ссылку),
    \begin{equation}\label{eq:18}
        \Big[f\big(x(t)\big)\Big]_t'(0) = 0 \iff f_x'(x_0) \cdot x_t'(0) = 0
    \end{equation}

    Касательное пространство к $N_{x_0}$ в точке $x_0$ имеет уравнение:
    \begin{equation}\label{eq:19}
        f_x'(x_0)\cdot \xi = 0
    \end{equation}

    Заметим, что \ref{eq:18} и \ref{eq:19} -- одно и то же уравнение, то есть
    \[
        x_t'(0) = \xi \implies x_t'(0) \in TN_{x_0}.
    \]
\end{proof}

\begin{theorem}[Необходимое условие условного локального экстремума]
    Пусть система уровнений
    \begin{equation}\label{eq:20}
        \left\{\begin{array}{l}
            F^1(x^1,\ldots,x^n) = 0 \\
            \vdots                  \\
            F^{n-k}(x^1,\ldots,x^n) = 0
        \end{array}\right.
    \end{equation}
    задает $(n-k)$-мерную гладкую поверхность $S$ в $D \subset \R^n, \ D$ -- область. Функция $f:D\rightarrow\R$ -- гладкая. Если $x_0 \in S$ является точкой условного локального экстремума для функции $f$, то существует такой набор чисел $\lambda_1,\lambda_2,\ldots,\lambda_{n-k} \in \R:$
    \[
        grad f(x_0) = \sum_{i = 1}^{n-k}\lambda_i \cdot grad F^i(x_0).
    \]
\end{theorem}

\begin{proof}
    Касательное пространство $TS_{x_0}$ задается системой уравнений:
    \begin{equation}\label{eq:21}
        \left\{\begin{array}{l}
            \frac{\delta F^1}{\delta x^1}(x_0)\cdot \xi^1 + \ldots + \frac{\delta F^1}{\delta x^n}(x_0) \cdot \xi^n = 0 \\
            \vdots                                                                                                      \\
            \frac{\delta F^{n-k}}{\delta x^1}(x_0)\cdot \xi^1 + \ldots + \frac{\delta F^{n-k}}{\delta x^n}(x_0) \cdot \xi^n = 0
        \end{array}\right.,
    \end{equation}
    но $\forall i = \overline{1,n-k}:$
    \[
        \left\{\frac{\delta F^i}{\delta x^1}\cdot (x_0),\ldots,\frac{\delta F^i}{\delta x^n}\cdot (x_0)\right\} = grad F^i(x_0).
    \]

    Перепишем \ref{eq:21} в виде:
    \begin{equation}\label{eq:22}
        \left\{\begin{array}{l}
            \big(grad F^1(x_0),\xi\big) = 0 \\
            \vdots                          \\
            \big(grad F^{n-k}(x_0,),\xi\big) = 0
        \end{array}\right.
    \end{equation}

    Касательное пространство $TN_{x_0}$ к $N_{x_0} = \big\{x \in D \ \big| \ f(x) = f(x_0)\big\}$ задается уравнением: $f'(x_0)\cdot\xi = 0$. Заметим, что:
    \[
        f'(x_0) = grad f(x_0) = \\ \left\{\frac{\delta f(x_0)}{\delta x^1},\ldots,\frac{\delta f(x_0)}{\delta x^n}\right\} \implies
    \]
    \begin{equation}\label{eq:23}
        \implies f'(x_0)\cdot \xi = 0 \iff \big(grad f(x_0),\xi\big) = 0
    \end{equation}

    Таким образом из леммы \ref{lemma:1} следует, что $\forall \xi$, удовлетворяющего системе уравнений \ref{eq:22}, так же удовлетворяет уравнению \ref{eq:23}, то есть из того, что $\forall i \in \overline{1,n-k}$
    \[
        \xi \perp grad F^i(x_0) \implies \xi \perp grad f(x_0) \implies
    \]
    $ \implies \exists \lambda_1,\ldots,\lambda_{n-k} \in \R $:
    \[
        grad f(x_0) = \sum_{i = 1}^{n-k} \lambda_i\cdot grad F^i(x_0).
    \]
\end{proof}

\newpage

\section*{Метод Лагранжа}

\begin{task}
    Пусть требуется найти условный экстремум функции $f:D\rightarrow\R, \ D$ -- область в $\R^n$, на поверхности $S$, заданной системой уравнений:
    \[
        \left\{\begin{array}{l}
            F^1(x^1,\ldots,x^n) = 0 \\
            \vdots                  \\
            F^k(x^1,\ldots,x^n) = 0
          \end{array}\right..
    \]
    
    Составим функцию Лагранжа:
    \begin{multline*}
      L(x,\lambda) = L(x^1,\ldots,x^n,\lambda^1,\ldots,\lambda^k) = \\
      = f(x^1,\ldots,x^n) + \sum_{i=1}^{k}\lambda^i\cdot F^i(x^1,\ldots,x^n),
    \end{multline*}
    
    $\lambda = (\lambda^1,\ldots,\lambda^k), \ \lambda^i \in \R$ -- коэффициент, в общем случае пока неизвестен.
    
    Необходимое условие локального экстремума для функции $L:$
    \begin{equation}\label{eq:24}
      \left\{\begin{array}{l}
        \left.\begin{array}{l}
            \frac{\delta L}{\delta x^1} = \frac{\delta f}{\delta x^1} + \sum_{i = 1}^{k}\lambda^i\cdot\frac{\delta F^i}{\delta x^1} = 0 \\
            \vdots                                                                                                                      \\
            \frac{\delta L}{\delta x^n} = \frac{\delta f}{\delta x^n} + \sum_{i = 1}^{k}\lambda^i\cdot\frac{\delta F^i}{\delta x^n} = 0 \\
        \end{array}\right\}\begin{array}{l}
            \text{необходимое условие условного}\\
            \text{экстремума функции }f
        \end{array} \\
        \left.\begin{array}{l}
            \frac{\delta L}{\delta \lambda^1} = F^1(x^1,\ldots,x^n) = 0                                                                 \\
            \vdots                                                                                                                      \\
            \frac{\delta L}{\delta \lambda^k} = F^k(x^1,\ldots,x^n) = 0
        \end{array}\right\}\text{ поверхность }S
      \end{array}\right.
    \end{equation}
\end{task}

\begin{definition}[Условный экстремум]
  Пусть $f:D\rightarrow\R, \ D \subset \R^n$ -- область, $S$ -- поверхность в $D$, \emph{условным экстремумом} функции $f$ называется экстремум функции $f\big|_S$.
\end{definition}