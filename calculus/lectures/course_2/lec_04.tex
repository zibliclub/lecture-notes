\lesson{4}{от 17 сен 2023 8:44}{Продолжение}


\begin{definition}[$ k $-мерная поверхность]
    Множество $ S \subset \R^n $ называется \emph{$ k $-мерной поверхностью}, если $ \forall x \in S \ \exists U(x)\subset\R^n $ и $ \exists $ диффиоморфизм $ \phi: U(x)\rightarrow I^n $:
    \[
        \phi\big(U(x)\cap S\big) = I^k,
    \] где $ I^n = \big\{x \in \R^n \ \big| \ |x^i| < 1 \big\} $,
    \[
        I^k = \big\{x \in \R^n \ \big| \ x^{k+1} = x^{k+2} = \ldots = x^n = 0\big\}.
    \]
\end{definition}

\newpage

\begin{example}
    $S = \R^n$ -- поверхность в $\R^n$,
    \[
        t^i(x^i) = \frac{2}{\pi}\cdot \arctan x^i,
    \]
    \begin{multline*}
        \mathfrak{I} = \left(\begin{matrix}
                \frac{\delta t^1}{\delta x^1} & \frac{\delta t^1}{\delta x^2} & \ldots & \frac{\delta t^1}{\delta x^n} \\
                \frac{\delta t^2}{\delta x^1} & \frac{\delta t^2}{\delta x^2} & \ldots & \frac{\delta t^2}{\delta x^n} \\
                \vdots                        & \vdots                        & \ddots & \vdots                        \\
                \frac{\delta t^n}{\delta x^1} & \frac{\delta t^n}{\delta x^2} & \ldots & \frac{\delta t^n}{\delta x^n} \\
            \end{matrix}\right) = \\
        = \left(\begin{matrix}
                \frac{2}{\pi} \cdot \frac{1}{1 + (x^1)^2} & 0                                         & \ldots & 0                                         \\
                0                                         & \frac{2}{\pi} \cdot \frac{1}{1 + (x^2)^2} & \ldots & 0                                         \\
                \vdots                                    & \vdots                                    & \ddots & \vdots                                    \\
                0                                         & 0                                         & \ldots & \frac{2}{\pi} \cdot \frac{1}{1 + (x^n)^2}
            \end{matrix}\right)
    \end{multline*}
\end{example}

\begin{statement}
    Пусть задана система уравнений:
    \begin{equation}\label{eq:7}
        \left\{\begin{array}{l}
            F^1(x^1,\ldots,x^n) = 0 \\
            \vdots                  \\
            F^{n-k}(x^1,\ldots,x^n) = 0
        \end{array}\right.,\quad F^i(x)\in C^{(1)}
    \end{equation}

    Кроме того,
    \[
        \left|\begin{matrix}
            \frac{\delta F^1}{\delta x^1}     & \ldots & \frac{\delta F^1}{\delta x^n}     \\
            \vdots                            & \ddots & \vdots                            \\
            \frac{\delta F^{n-k}}{\delta x^1} & \ldots & \frac{\delta F^{n-k}}{\delta x^n}
        \end{matrix}\right|(x) \ne 0, \quad \forall x \in \R^n
    \]

    Тогда решение этой системы является $k$-мерной поверхностью в $\R^n$.
\end{statement}

\begin{proof}
    По теореме о неявной функции, система \ref{eq:7} эквивалентна системе
    \[
        \left\{\begin{array}{l}
            x^{k+1} = f^1(x^1,\ldots,x^k) \\
            x^{k+2} = f^2(x^1,\ldots,x^k) \\
            \vdots                        \\
            x^n = f^{n-k}(x^1,\ldots,x^k)
        \end{array}\right.
    \]

    Положим:
    \begin{align*}
        t^1     & = x^1                               \\
        t^2     & = x^2                               \\
        \vdots                                        \\
        t^k     & = x^k                               \\
        t^{k+1} & = x^{k+1} - f^1(x^1,\ldots,x^k) = 0 \\
        t^{k+2} & = x^{k+2} - f^2(x^1,\ldots,x^k) = 0 \\
                & \vdots                              \\
        t^{n}   & = x^n - f^{n-k}(x^1,\ldots,x^k) = 0
    \end{align*}

    Таким образом построенное отображение является диффиоморфизмом $\implies$ решение системы \ref{eq:7} -- $k$-мерная поверхность в $\R^n$.
\end{proof}

\begin{definition}[Локальная карта или параметризация поверхности]
    Пусть $S$ -- $k$-мерная поверхность в $\R^n, \ x_0 \in S$ и $\phi: U(x_0) \rightarrow I^n$ -- диффиоморфизм:
    \[
        \phi\big(U(x_0)\cap S\big) = I^k.
    \]

    Ограничение $\phi^{-1}$ на $I^k$ будем называть \emph{локальной картой} или \emph{параметризацией поверхности} $S$ в окрестности точки $x_0$.
\end{definition}

\begin{definition}[Касательное пространство]
    Пусть $ S $ -- $ k $-мерная поверхность в $ \R^n, \ x_0 \in S, \ x = x(t):\R^k \rightarrow \R^n $ -- параметризация $ S $ в окрестности точки $ x_0 $, при этом $ x_0 = x(0) $.

    \emph{Касательным пространством} (или плоскостью) к $S$ в точке $x_0$ называется $k$-мерная плоскость, заданная уравнением:
    \begin{equation}\label{eq:8}
        x = x_0 + x'(0) \cdot t,
    \end{equation}
    \begin{align*}
        x_0   & = (x^1_0,x^2_0,\ldots,x^n_0)                                                   \\
        x(t)  & = \left\{\begin{array}{l}
                             x^1(t^1,\ldots,t^k) \\
                             x^2(t^1,\ldots,t^k) \\
                             \vdots              \\
                             x^n(t^1,\ldots,t^k)
                         \end{array}\right.                                                   \\
        x'(t) & = \left(\begin{matrix}
                                \frac{\delta x^1}{\delta t^1} & \ldots & \frac{\delta x^1}{\delta t^k} \\
                                \vdots                        & \ddots & \vdots                        \\
                                \frac{\delta x^n}{\delta t^1} & \ldots & \frac{\delta x^n}{\delta t^k} \\
                            \end{matrix}\right)(t), \quad t = \left(\begin{matrix}
                                                                        t^1 \\ t^2 \\ \vdots \\ t^k
                                                                    \end{matrix}\right)
    \end{align*}

    Таким образом касательное пространство задается системой из \ref{eq:8}:
    \[
        \left\{\begin{array}{l}
            x^1 = x^1_0 + \frac{\delta x^1}{\delta t^1} (0)\cdot t^1 + \ldots + \frac{\delta x^1}{\delta t^k}(0)\cdot t^k \\
            x^2 = x^2_0 + \frac{\delta x^2}{\delta t^1} (0)\cdot t^1 + \ldots + \frac{\delta x^2}{\delta t^k}(0)\cdot t^k \\
            \vdots                                                                                                        \\
            x^n = x^n_0 + \frac{\delta x^n}{\delta t^1} (0)\cdot t^1 + \ldots + \frac{\delta x^n}{\delta t^k}(0)\cdot t^k
        \end{array}\right..
    \]
\end{definition}

\begin{example}
    Пусть $\gamma = \gamma (t)$ -- гладкая кривая в $\R^3, \ \gamma:\left\{\begin{array}{c}
            x = x(t) \\
            y = y(t) \\
            z = z(t)
        \end{array}\right.$.

    Обозначим $x_0 = x(0), \ y_0 = y(0), \ z_0 = z(0)$.

    \ref{eq:8} -- касательное пространство к кривой $\gamma$ в точке $x_0$.
    \[
        \left\{\begin{array}{l}
            x = x_0 + x'(0) \cdot t \\
            y = y_0 + y'(0) \cdot t \\
            z = z_0 + z'(0) \cdot t \\
        \end{array}\right. \implies \left\{\begin{array}{l}
            x - x_0 = x'(0)\cdot t \\
            y - y_0 = y'(0)\cdot t \\
            z - z_0 = z'(0)\cdot t \\
        \end{array}\right. \implies
    \]
    \[
        \implies \frac{x - x_0}{x'(0)} = \frac{y - y_0}{y'(0)} = \frac{z - z_0}{z'(0)} = t.
    \]
\end{example}

\begin{example}
    $x^2 + y^2 + z^2 = 1$

    Пусть $z_0 > 0$, тогда в окрестности точки $(x_0,y_0,z_0)$ сферу можно параметризировать следующими уравнениями:
    \[
        \left\{\begin{array}{l}
            x = u \\
            y = v \\
            z = \sqrt{1 - u^2 - v^2}
        \end{array}\right..
    \]

    Касательное пространство к сфере в точке $(x_0,y_0,z_0)$:
    \[
        \left(\begin{matrix}
                x \\
                y \\
                z
            \end{matrix}\right) = \left(\begin{matrix}
                x_0 \\
                y_0 \\
                z_0
            \end{matrix}\right) + \left(\begin{matrix}
                \frac{\delta x}{\delta u} & \frac{\delta x}{\delta v} \\
                \frac{\delta y}{\delta u} & \frac{\delta y}{\delta v} \\
                \frac{\delta z}{\delta u} & \frac{\delta z}{\delta v} \\
            \end{matrix}\right) (u_0,v_0) \cdot \left(\begin{matrix}
                u \\
                v
            \end{matrix}\right) \implies
    \]
    \[
        \implies \left\{\begin{array}{l}
            x = x_0 + \frac{\delta x}{\delta u}(u_0,v_0) \cdot u + \frac{\delta x}{\delta v}(u_0,v_0)\cdot v \\
            y = y_0 + \frac{\delta y}{\delta u}(u_0,v_0) \cdot u + \frac{\delta y}{\delta v}(u_0,v_0)\cdot v \\
            z = z_0 + \frac{\delta z}{\delta u}(u_0,v_0) \cdot u + \frac{\delta z}{\delta v}(u_0,v_0)\cdot v \\
        \end{array}\right. \implies
    \]
    \[
        \implies \left\{\begin{array}{l}
            x = x_0 + u \\
            y = y_0 + v \\
            z = z_0 - \frac{u_0}{\sqrt{1 - u_0^2 - v_0^2}}\cdot u - \frac{v_0}{\sqrt{1 - u_0^2 - v_0^2}}\cdot v
        \end{array}\right..
    \]
\end{example}

\newpage

\begin{statement}
    Пусть $S$ -- $k$-мерная поверхность в $\R^n$ задается системой уравнений:
    \begin{equation}\label{eq:9}
        \left\{\begin{array}{l}
            F^1(x^1,\ldots,x^n) = 0 \\
            \vdots                  \\
            F^{n-k}(x^1,\ldots,x^n) = 0,
        \end{array}\right.\text{, причем }\left|\begin{array}{ccc}
            \frac{\delta F^1}{\delta x^{k+1}}     & \ldots & \frac{\delta F^1}{\delta x^n}     \\
            \vdots                                & \ddots & \vdots                            \\
            \frac{\delta F^{n-1}}{\delta x^{k+1}} & \ldots & \frac{\delta F^{n-1}}{\delta x^n} \\
        \end{array}\right|(x_0) \ne 0
    \end{equation}

    Тогда касательная плоскость к $S$ в точке $x_0$ задается системой уравнений:
    \[
        \left\{\begin{array}{l}
            \frac{\delta F^1}{\delta x^1}(x_0)\cdot (x^1-x^1_0) + \ldots + \frac{\delta F^1}{\delta x^n}(x_0)(x^n - x_0^n) = 0 \\
            \vdots                                                                                                             \\
            \frac{\delta F^{n-k}}{\delta x^1}(x_0)\cdot (x^1-x^1_0) + \ldots + \frac{\delta F^{n-k}}{\delta x^n}(x_0)(x^n - x_0^n) = 0
        \end{array}\right.,
    \]
    или кратко:
    \[
        F'(x_0)\cdot(x-x_0) = 0.
    \]
\end{statement}

\begin{proof}
    Обозначим $(x^1,\ldots,x^k) = u, \ (x^{k+1},\ldots,x^n) = v$,
    \[
        F = \left(\begin{matrix}
                F^1    \\
                \vdots \\
                F^{n-k}
            \end{matrix}\right).
    \]

    Тогда условия утверждения запишем в виде:
    \begin{equation*}
        F(u,v) = 0, \quad \big|F'_v(u_0,v_0)\big| \ne 0.
    \end{equation*}

    Тогда по теореме о неявной функции система \ref{eq:9} эквивалентна системе
    \[
        \left\{\begin{array}{l}
            u = u \\
            v = f(u)
        \end{array}\right..
    \]

    Тогда касательная плоскость задается:
    \[
        \text{роль }t = \left(\begin{array}{c}
                t^1    \\
                \vdots \\
                t^k
            \end{array}\right)\text{ играет } u = \left(\begin{array}{c}
                x^1    \\
                \vdots \\
                x^k
            \end{array}\right).
    \]

    Тогда систему можно записать в виде:
    \[
        \left\{\begin{array}{l}
            x^1 = t^1                       \\
            \vdots                          \\
            x^k = t^k                       \\
            x^{k + 1} = f^1(t^1,\ldots,t^k) \\
            \vdots                          \\
            x^n = f^{n-k}(t^1,\ldots,t^k)
        \end{array}\right.,
    \]
    \[
        t_0 = (t_0^k,\ldots,t_0^k) = (x_0^1,\ldots,x_0^k).
    \]

    \[
        x'(t_0) = \left(\begin{matrix}
                \frac{\delta x^1}{\delta t^1}     & \ldots & \frac{\delta x^1}{\delta t^k}     \\
                \vdots                            & \ddots & \vdots                            \\
                \frac{\delta x^k}{\delta t^1}     & \ldots & \frac{\delta x^k}{\delta t^k}     \\
                \frac{\delta f^1}{\delta t^1}     & \ldots & \frac{\delta f^1}{\delta t^k}     \\
                \vdots                            & \ddots & \vdots                            \\
                \frac{\delta f^{n-k}}{\delta t^1} & \ldots & \frac{\delta f^{n-k}}{\delta t^k}
            \end{matrix}\right)(t_0),
    \]
    \[
        x = x_0 + x'(t_0)\cdot t.
    \]
    \begin{equation}\label{eq:10}
        \left\{\begin{array}{l}
            x^1 = x_0^1 + 1 \cdot t^1                                                                                                 \\
            \vdots                                                                                                                    \\
            x^k = x_0^k + 1 \cdot t^k                                                                                                 \\
            x^{k+1} = x_0^{k+1} + \frac{\delta f^1}{\delta t^1}(t_0) \cdot t^1 + \ldots + \frac{\delta f^1}{\delta t^k}(x_0)\cdot t^k \\
            \vdots                                                                                                                    \\
            x^n = x_0^n + \frac{\delta f^{n-k}}{\delta t^1}(t_0)\cdot t^1 + \ldots + \frac{\delta f^{n-k}}{\delta t^k}(t_0)\cdot t^k
        \end{array}\right.
    \end{equation}

    Из \ref{eq:10} следует, что:
    \begin{align*}
         & t^1 = x^1 - x_0^1, \\
         & t^2 = x^2 - x_0^2, \\
         & \vdots             \\
         & t^k = x^k - x_0^k.
    \end{align*}

    Из теоремы \ref{theorem:1}:
    \[
        f'(u_0) = -\big[F_v'(u_0,v_0)\big]^{-1} \cdot F_u'(u_0,v_0).
    \]

    Преобразуем \ref{eq:10}:
    \[
        \left\{\begin{array}{l}
            \left.\begin{array}{l}
                      x^1 - x_0^1 = t^1 \\
                      \vdots            \\
                      x^k - x_0^k = t^k
                  \end{array}\right\} \ u - u_0 = u - u_0                                                                                                \\
            x^{k+1} - x_0^{k+1} = \frac{\delta f^1}{\delta t^1}(t_0)\cdot (x^1 - x_0^1) + \ldots + \frac{\delta f^1}{\delta t^k}(t_0)\cdot (x^k - x_0^k) \\
            \vdots                                                                                                                                       \\
            x^n - x_0^n = \frac{\delta f^{n-k}}{\delta t^1}(t_0)\cdot (x^1 - x_0^1) + \ldots + \frac{\delta f^{n-k}}{\delta t^k}(t_0)\cdot (x^k - x_0^k)
        \end{array}\right.
    \]
    \[
        \left\{\begin{array}{l}
            u-u_0 = u-u_0 \\
            v-v_0 = f'(u_0) \cdot (u-u_0)
        \end{array}\right.
    \]
    \[
        \left\{\begin{array}{l}
            u-u_0 = u-u_0 \\
            v - v_0 = \underbrace{-\big[F_v'(u_0,v_0)\big]^{-1}\cdot F_u'(u_0,v_0)\cdot (u-u_0)}_{f'(u_0)} \quad \Big| \cdot F_v'(u_0,v_0)
        \end{array}\right.
    \]
    $ \implies \big[F_v'(u_0,v_0)\big](v-v_0) + F_u'(u_0,v_0)\cdot(u-u_0) = 0 $.
\end{proof}

\begin{note}
    Итак, мы вывели, что если поверхность задана системой уравнений
    \[
        \left\{\begin{array}{l}
            F^1(x^1,\ldots,x^n) = 0 \\
            \vdots                  \\
            F^{n-k}(x^1,\ldots,x^n) = 0
        \end{array}\right.\text{ или }F(x) = 0,
    \]
    \[
        F = \left(\begin{matrix}
                F^1(x) \\
                \vdots \\
                F^{n-k}(x)
            \end{matrix}\right), \begin{array}{l}
            x = (x^1,\ldots,x^n) \\
            x_0 = (x_0^1,\ldots,x_0^n)
        \end{array}.
    \]

    Тогда уравнение касательной плоскости кратко записывается:
    \[
        F_x'(x_0)\cdot(x-x_0) = 0.
    \]

    Обозначим $x - x_0 = \xi$, то есть:
    \[
        \xi = \left(\begin{matrix}
                \xi^1  \\
                \vdots \\
                \xi^n
            \end{matrix}\right) = \left(\begin{matrix}
                x'-x_0' \\
                \vdots  \\
                x^n-x_0^n
            \end{matrix}\right).
    \]

    Таким образом получаем, что уравнение касательного пространства (плоскости) имеет вид:
    \[
        F_x'(x_0)\cdot \xi = 0.
    \]

    Таким образом касательнаое пространство (плоскость) к поверхности, заданной уравнением $F(x) = 0$, в точке $x_0$ состоит из векторов $\xi$, удовлетворяющих уравнению:
    \begin{equation}\label{eq:11}
        F_x'(x_0)\cdot \xi = 0
    \end{equation}
\end{note}