\section{Логика предикатов}

\subsection{Понятие предиката и операции, их представления, примеры.}

\begin{definition}[$ n $-мерный предикат]
    \emph{$ n $-мерный предикат} на множестве $ A $ -- это отображение вида:
    \[
        P: A^n \rightarrow\{0,1\}.
    \]

    При этом $ n $-местный $ P $.

    Неформально, \emph{предикат} -- это высказывание, зависящее от параметров.
\end{definition}

\begin{example}
    $ A = \Z $
    \[
        \begin{array}{lll}
            P(x) = 1     & \iff & x \text{ простое число} \\
            Q(x,y) = 1   & \iff & x + y = 4               \\
            R(x,y) = 1   & \iff & x < y                   \\
            T(x,y,z) = 1 & \iff & z = \text{НОД}(x,y)
        \end{array}
    \]
\end{example}

\begin{example}
    $ A $ -- множество людей.

    Примеры предикатов на $ A $:
    \[
        \begin{array}{lll}
            P(x) = 1   & \iff & x \text{ -- женщина}  \\
            Q(x,y) = 1 & \iff & x \text{ родитель } y \\
            R(x,y) = 1 & \iff & x,y \text{ -- братья}
        \end{array}
    \]
\end{example}

\begin{definition}[$ n $-местная операция]
    \emph{$ n $-местная операция} на множестве $ A $ -- это отображение вида:
    \[
        f: A^n \rightarrow A.
    \]
\end{definition}

\begin{example}
    $ A = \Z $
    \[
        \begin{array}{ll}
            f_1(x) = x+1 & g_1(x,y) = \left\{\begin{array}{rl}
                                                 x^y, & y > 0        \\
                                                 0    & \text{иначе}
                                             \end{array}\right.                \\
            f_2(x) = 2x  & g_2(x,y) = x+y                                       \\
            f_3(x) = 0   & g_3(x,y) = \text{сумма последних цифр }x \text{ и }y \\
            f_4(x) = x^2 &
        \end{array}
    \]
    \[
        \forall x \ \Big(P(x)\land Q(x) \rightarrow R\big(f(x)\big)\Big).
    \]
\end{example}

\begin{remark}
    Чтобы писать формулы, достаточно иметь только обозначения предикатов и операций и знать их местности.
\end{remark}

\subsection{Сигнатура, интерпретация сигнатуры на множестве, алгебраические системы.}

\begin{note}
    \emph{Сигнатура} -- это набор предикатных, функциональных и константных символов с указанием их местностей.
\end{note}

\begin{definition}[Сигнатура]
    \emph{Сигнатура} -- набор трех непересекающихся множеств:
    \[
        \mathcal{P} \cup \mathcal{F} \cup \mathcal{C},
    \]
    где элементы множества $ \mathcal{P} $ назовем \emph{предикатные символы}, элементы $ \mathcal{F} $ -- \emph{функциональные символы}, элементы $ \mathcal{C} $ -- \emph{константные} символы.

    Так же должна быть определена функция:
    \[
        \mu : \mathcal{P} \cup \mathcal{F} \rightarrow \N \text{ -- местность символов}.
    \]
\end{definition}

\begin{example}
    \[
        \sigma = \{\underbrace{P^{(1)}, Q^{(2)}}_{\text{предикат.}}, \underbrace{f^{(1)}, g^{(2)}}_{\text{функц.}}, \underbrace{c}_{\text{конст.}}\}
    \]
\end{example}

\begin{definition}[Интерпретация сигнатуры на множестве]
    \emph{Интерпретация сигнатуры} $ \sigma $ \emph{на множестве} $ A $ -- это отображение $ I $, которое
    \begin{itemize}
        \item каждый предикатный символ $ P^{(n)}\in\sigma $ отображает в $ n $-местный предикат на множестве $ A $,
        \item каждый функциональный символ $ f^{(n)}\in\sigma $ отображает в $ n $-местную операцию на $ A $,
        \item каждый константный символ отображает в элемент множества $ A $.
    \end{itemize}
\end{definition}

\begin{definition}[Алгебраическая система]
    \emph{Алгебраическая система} -- это набор, состоящий из множества $ A $, сигнатуры $ \sigma $ и интерпретации сигнатуры $ \sigma $ на множестве $ A $.

    Множество $ A $ -- основное множество системы.

    \[
        \mathfrak{N} = <A,\sigma,\underbrace{I}_{\begin{matrix}
                \text{\scriptsize часто не} \\
                \text{\scriptsize пишут}
            \end{matrix}}>.
    \]
\end{definition}

\begin{example}
    $ \sigma = \{P^{(1)},Q^{(2)},f^{(1)},g^{(2)},a,b\} $

    $ A = \Z $, интерпретация:
    \[
        \begin{array}{l}
            P(x) = 1 \iff x > 0                            \\
            Q(x,y) = 1 \iff x,y \text{ -- взаимно простые} \\
            f(x) = x+1                                     \\
            g(x,y) = xy + 1                                \\
            a = 0, \ b = 1
        \end{array}
    \]
    \[
        \mathfrak{N} = <A,\sigma>.
    \]
\end{example}

\subsection{Язык логики предикатов, термы, формулы логики предикатов.}

\begin{note}
    Пусть $ \sigma $ -- сигнатура.

    \emph{Алфавит языка логики предикатов} сигнатуры $ \sigma $ -- это:
    \[
        \mathfrak{A}_\sigma = \sigma \cup \{x_1,x_2,\ldots,\land,\lor,\lnot,\rightarrow,\leftrightarrow, (,),\forall,\exists, ,\}.
    \]
\end{note}

\begin{definition}[Терм]
    \emph{Терм} сигнатуры $ \sigma $ -- это слово, построенное по правилам:
    \begin{enumerate}
        \item Символ переменной -- терм.
        \item Константный символ сигнатуры $ \sigma $ -- это терм.
        \item Если $ f^{(n)} \in \sigma $ -- функциональный символ, $ t_1,\ldots,t_n $ -- термы, то слово $ f(t_1,\ldots,t_n) $ -- тоже терм.
    \end{enumerate}
\end{definition}

\begin{definition}[Атомарная формула]
    \emph{Атомарная формула} сигнатуры $ \sigma $ -- это слово одного из двух видов:
    \begin{enumerate}
        \item $ t_1 = t_2 $, где $ t_1,t_2 $ -- термы.
        \item $ P(t_1,\ldots,t_n) $, где $ P^{(n)} \in \sigma $ -- предикатный символ, $ t_1,\ldots,t_n $ -- термы.
    \end{enumerate}
\end{definition}

\begin{definition}[Формула языка логики предикатов]
    \emph{Формула языка логики предикатов} сигнатуры $ \sigma $ -- это слово, построенное по правилам:
    \begin{enumerate}
        \item Атомарная формула -- это формула.
        \item Если $ \phi_1 $ и $ \phi_2 $ -- формулы, то слова $ (\phi_1 \land \phi_2),(\phi_1 \lor \phi_2), (\phi_1 \rightarrow \phi_2),(\phi_1 \leftrightarrow \phi_2), \lnot\phi_1 $ -- тоже формулы.
        \item Если $ \phi $ -- формула, $ x $ -- переменная, то слова $ (\forall x \phi) $ и $ (\exists x \phi) $ -- тоже формулы.
    \end{enumerate}
\end{definition}

\subsection{Свободные и связанные переменные. Замкнутые формулы.}

\begin{definition}[Сявзанные, свободные переменные]
    Вхождение переменной $ x $ в формулы вида
    \[
        (\forall x \ \phi) \text{ и } (\exists x \ \phi)
    \]
    назовем \emph{связанным}.

    В противном случае, вхождение переменной \emph{свободное}.
    \[
        P(\underbrace{x}_{\begin{matrix}
                \text{\scriptsize своб.} \\
                \text{\scriptsize вх.}
            \end{matrix}}) \cup \forall x \ Q(\underbrace{x}_{\begin{matrix}
                \text{\scriptsize связ.} \\
                \text{\scriptsize вх.}
            \end{matrix}},\underbrace{y}_{\begin{matrix}
                \text{\scriptsize своб.} \\
                \text{\scriptsize вх.}
            \end{matrix}}).
    \]
\end{definition}

\begin{note}
    Переменная $ x $ \emph{свободная} в формуле $ \phi $, если есть хотя бы одно ее свободное вхождение в $ \phi $. В противном случае -- переменная \emph{связная}.
\end{note}

\begin{definition}[Замкнутая формула]
    \emph{Замкнутая формула (предложение)} -- это формула без свободных переменных.
\end{definition}

\subsection{Истинность формул на алгебраической системе.}

\begin{definition}[Истинность формулы на алгебраической системе]
    Пусть $ \phi(x_1,\ldots,x_n) $ -- формула, $ \mathfrak{N} = <A,\sigma> $ -- алгебраическая система,\\$ a_1,\ldots,a_n \in A $.

        \emph{Истинность формулы} $ \phi $ \emph{на алгебраической системе} $ \mathfrak{N} $ на элементах $ a_1,\ldots,a_n \ \big(\mathfrak{N} \vDash \phi(a_1,\ldots,a_n)\big) $ определяется по следующим правилам:
    \begin{enumerate}
        \item Пусть $ \phi $ имеет вид $ t_1(x_1,\ldots,x_n) = t_2(x_1,\ldots,x_n) $, где $ t_1,t_2 $ -- термы. Тогда:
              \[
                  \mathfrak{N} \vDash \phi(a_1,\ldots,a_n)\iff t_{1_\mathfrak{N}}(a_1,\ldots,a_n) = t_{2_\mathfrak{N}} (a_1,\ldots,a_n).
              \]

        \item Пусть $ \phi $ имеет вид $ P(t_1,\ldots,t_k) $, где $ P^{(k)}\in \sigma $ -- предикатный символ, $ t_1(x_1,\ldots,x_n),\ldots,t_k(x_1,\ldots,x_n) $ -- термы. Тогда:
              \[
                  \mathfrak{N}\vDash \phi(a_1,\ldots,a_n) \iff P_\mathfrak{N}\big(\underbrace{t_{1_\mathfrak{N}}(a_1,\ldots,a_n)}_{b_1},\ldots,\underbrace{t_{k_\mathfrak{N}}(a_1,\ldots,a_n)}_{b_k}\big) = 1.
              \]

        \item Пусть $ \phi = (\phi_1 \land \phi_2) \ \big((\phi_1 \lor \phi_2),(\phi_1 \rightarrow \phi_2),(\phi_1 \leftrightarrow \phi_2),\lnot\phi_2\big) $. Истинность формулы $ \phi $ определяется из значений формул $ \phi_1(a_1,\ldots,a_n) $ и $ \phi_2(a_1,\ldots,a_n) $ по таблицам истинности для логических связок.

        \item Пусть $ \phi(x_1,\ldots,x_n) $ имеет вид $ \exists x \ \psi (x,x_1,\ldots,x_n) $.

              Тогда $ \mathfrak{N} \vDash \phi (a_1,\ldots,a_n) \iff $ для некоторого $ b \in A \ \mathfrak{N} \vDash \psi(b,a_1,\ldots,a_n) $.

        \item Пусть $ \phi $ имеет вид $ \forall x \ \psi(x,x_1,\ldots,x_n) $.

              Тогда $ \mathfrak{N} \vDash \phi(a_1,\ldots,a_n) \iff $ для всех $ b\in A \ \mathfrak{N}\vDash\psi(b,a_1,\ldots,a_n) $.
    \end{enumerate}
\end{definition}

\subsection{Выразимость свойств в логике предикатов. Умение записать формулой различные свойства систем и элементов систем.}

\begin{note}
    Пусть $ \phi(x_1,\ldots,x_n) $ -- формула сигнатуры $ \sigma, \ \mathfrak{N} = <A,\sigma> $ -- алгебраическая система.

    Множество истинности формулы $ \phi $ в алгебраической системе $ \mathfrak{N} $ -- это:
    \[
        A_\phi = \big\{(a_1,\ldots,a_n) \ \big| \ a_i \in A, \ \mathfrak{N} \vDash \phi(a_1,\ldots,a_n)\big\}.
    \]
\end{note}

\begin{definition}[Выразимое множество в алгебраической системе]
    Множество $ B \subseteq A^n $ \emph{выразимо в алгебраической системе} $ \mathfrak{N} $, если существует формула $ \phi(x_1,\ldots,x_n) $ сигнатуры $ \sigma $, для которой $ B = A_\phi $ ($ B $ выразимо в $ \mathfrak{N} \iff \exists \phi(x_1,\ldots,x_n): \ \forall a_1,\ldots,a_n \ (a_1,\ldots,a_n) \in B \iff \mathfrak{N}\vDash \phi(a_1,\ldots,a_n) $)
\end{definition}

\begin{definition}[Выразимая функция в алгебраической системе]
    Функция $ f: A^n \rightarrow A $ \emph{выразима} в $ \mathfrak{N} $, если существует формула $ \phi(x_1,\ldots,x_n,y) $ такая, что $ \forall a_1,\ldots,a_n,b \in A \ b = f(a_1,\ldots,a_n) \iff \mathfrak{N}\vDash (a_1,\ldots,a_n,b) $.
\end{definition}

\subsection{Тождественно истинные (ложные) и выполнимые формулы.}

\begin{definition}[Тождественно истинная (ложная), выполнимая формула в системе]
    Формула $ \phi(x_1,\ldots,x_n) $ \emph{тождественно истинна (ложна) в системе} $ \mathfrak{N} $, если $ \forall a_1,\ldots,a_n \in \mathfrak{N} $
    \[
        \mathfrak{N} \vDash \phi(a_1,\ldots,a_n) \quad \big(\mathfrak{N}\nvDash \phi(a_1,\ldots,a_n)\big).
    \]

    Формула $ \phi $ \emph{выполнима в системе} $ \mathfrak{N} $, если $ \exists a_1,\ldots,a_n $
    \[
        \mathfrak{N} \vDash \phi(a_1,\ldots,a_n).
    \]
\end{definition}

\begin{definition}[Тождественно истинная (ложная), выполнимая формула]
    Формула $ \phi $ \emph{тождественно истинная (ложная)}, если $ \phi $ тождественно истинная (ложная) в любой системе сигнатуры $ \sigma $.

    Формула $ \phi $ \emph{выполнимая}, если она выполнимая хотя бы в одной алгебраческой системе сигнатуры $ \sigma $.
\end{definition}

\newpage

\subsection{Эквивалентность формул логики предикатов.}

\begin{definition}[Эквивалентные формулы в алгебраической системе]
    Формулы $ \phi(x_1,\ldots,x_n) $ и $ \psi(x_1,\ldots,x_n) $ сигнатуры $ \sigma $ \emph{эквивалентны} \\ \emph{в алгебраической системе} $ \mathfrak{N} = <A,\sigma> $, если $ \forall a_1,\ldots,a_n \in A $
    \[
        \mathfrak{N} \vDash \phi(a_1,\ldots,a_n) \iff \mathfrak{N} \vDash \psi(a_1,\ldots,a_n).
    \]
    \[
        \text{Обозначение:} \quad \phi \sim_\mathfrak{N} \psi.
    \]
\end{definition}

\begin{definition}[Эквивалентные формулы]
    Формулы $ \phi $ и $ \psi $ сигнатуры $ \sigma $ \emph{эквивалентны}, если они эквивалентны в любой алгебраической системе сигнатуры $ \sigma $.
    \[
        \text{Обозначение:} \quad \phi \sim \psi.
    \]
\end{definition}

\subsection{Основные эквивалентности логики предикатов.}

\begin{theorem}[Основные эквивалентности логики предикатов]
    Справедливы следующие эквивалентности:
    \[
        \begin{array}{l}
            \forall x \forall y \ \phi(x,y) \sim \forall y \forall x \ \phi(x,y) \\
            \exists x \exists y \ \phi(x,y) \sim \exists y \exists x \ \phi(x,y)
        \end{array}
    \]
    \[
        \begin{array}{l}
            \forall x \ \phi(x) \sim \forall y \ \phi(y), \quad y\text{ не входит свободно в }\phi(x) \\
            \exists x \ \phi(x) \sim \exists y \ \phi(y), \quad x\text{ не входит свободно в }\phi(y)
        \end{array}
    \]
    \[
        \begin{array}{l}
            \lnot \forall x \ \phi(x) \sim \exists x \ \lnot \phi(x) \\
            \lnot \exists x \ \phi(x) \sim \forall x \ \lnot \phi(x)
        \end{array}
    \]
    \[
        \begin{array}{l}
            \big(\forall x \ \phi(x)\big) \land \big(\forall x \ \psi(x)\big) \sim \forall x \ \big(\phi(x)\land \psi(x)\big) \\
            \big(\exists x \ \phi(x)\big) \lor \big(\exists x \ \psi(x)\big) \sim \exists x \ \big(\phi(x)\lor \psi(x)\big)
        \end{array}
    \]
    \[
        \begin{array}{l}
            \big(\forall x \ \phi(x)\big) \lor \big(\forall y \ \psi(y)\big) \sim \forall x \forall y \ \big(\phi(x)\lor \psi(y)\big) \\
            \big(\exists x \ \phi(x)\big) \land \big(\exists y \ \psi(y)\big) \sim \exists x \exists y \ \big(\phi(x)\land \psi(y)\big)
        \end{array}
    \]
    \[
        \begin{array}{l}
            \big(\forall x \ \phi(x)\big) \land \big(\exists y \ \psi(y)\big) \sim \forall x \exists y \ \big(\phi(x) \land \psi(y)\big) \sim \exists y \forall x \ \big(\phi(x) \land \psi(y)\big) \\
            \big(\forall x \ \phi(x)\big) \lor \big(\exists y \ \psi(y)\big) \sim \forall x \exists y \ \big(\phi(x) \lor \psi(y)\big) \sim \exists y \forall x \ \big(\phi(x) \lor \psi(y)\big)
        \end{array}
    \]
    \[
        \begin{array}{l}
            \left.\begin{array}{l}
                      \big(\forall x \ \phi(x)\big)\land\psi \sim \forall x \ \big(\phi(x)\land \psi\big) \\
                      \big(\forall x \ \phi(x)\big)\lor\psi \sim \forall x \ \big(\phi(x)\lor \psi\big)   \\
                      \big(\exists x \ \phi(x)\big)\land\psi \sim \exists x \ \big(\phi(x)\land \psi\big) \\
                      \big(\exists x \ \phi(x)\big)\lor\psi \sim \exists x \ \big(\phi(x)\lor \psi\big)
                  \end{array}\right\} \quad x \text{ не входит свободно в }\psi
        \end{array}
    \]
\end{theorem}

\subsection{Пренексный вид формулы.}

\begin{definition}[Пренексный вид формулы]
    Формула $ \phi $ находится в \emph{пренексном виде} (предваренная нормальная форма), если
    \begin{itemize}
        \item либо $ \phi $ кванторов не содержит ($ \phi $ бескванторная), \\
        \item либо $ \phi $ имеет вид $ (Q_1x_1)\ldots(Q_nx_n) \ \psi(x_1,\ldots,x_n) $,
    \end{itemize}
    где $ Q_i $ -- кванторы $ \forall $ или $ \exists $, $ \psi $ -- бескванторная.
\end{definition}

\begin{theorem}[О пренексном виде формулы]
    Любую формулу логики предикатов можно преобразовать в эквивалентную формулу в пренексном виде.
\end{theorem}

\begin{note}[Алгоритм приведения]\leavevmode
    \begin{enumerate}
        \item Выразить все логические связки через $ \land,\lor,\lnot $.
        \item Переименовать все связанные переменные так, чтобы они отличались от свободных и друг от друга.
        \item Двигаясь от самых внутренних подформул наружу, применяя стандартные эквивалентности, выносим все кванторы влево.
    \end{enumerate}
\end{note}

\begin{example}
    \begin{multline*}
        \Big[\forall x \ \big(\exists y \ Q(x,y)\big) \land \big(\forall y \ P(y)\big)\Big] \rightarrow \Big[\big(\forall x \exists y \exists z \ P(x,y,z)\big) \lor \bigl(\exists x \forall y \ Q(x,y)\bigr)\Big] \sim \\
        \sim \lnot\Big[\forall x \Big(\big(\exists y \ Q(x,y)\big) \land \forall y \ P(y)\Big)\Big] \lor \Big[\bigl(\forall x \exists y \exists z P(x,y,z)\bigr) \lor\bigl(\exists x \forall y \ Q(x,y)\bigr)\Big] \sim \\
        \sim \Big[\forall x \Bigl(\bigl(\exists y \ Q(x,y)\bigr)\land \forall t \ P(t)\Bigr)\Big] \lor \Big[\bigl(\forall u \exists v \exists z \ r(u,v,z)\bigr) \lor\bigl(\exists\omega \forall k \ Q(\omega,k)\bigr)\Big].
    \end{multline*}
\end{example}

\subsection{Классы формул $\Sigma_n$, $\Pi_n$, $\Delta_n$. Теорема о связях между этими классами.}

\begin{definition}[Классы формул $\Sigma_n$, $\Pi_n$, $\Delta_n$]
    Для $ n > 0 $:
    \begin{itemize}
        \item класс $ \Sigma_n $ состоит из всех формул в пренексном виде, у которых кванторный префикс начинается с квантора $ \exists $ и содержит $ n-1 $ перемену кванторов,
        \item класс $ \Pi_n $ состоит из всех формул в пренексном виде, у которых кванторный префикс начинается с квантора $ \forall $ и содержит $ n-1 $ перемену кванторов,
        \item класс $ \Delta_n $ состоит из всех формул, которые можно привести и к виду $ \Sigma_n $, и к виду $ \Pi_n $,
        \item при $ n=0 $:
              \[
                  \Sigma_0 = \Pi_0 = \Delta_0 \text{ -- все бескванторные формулы}.
              \]
    \end{itemize}

    Стрелка $ K_1 \rightarrow K_2 $ означает, что все формулы класса $ K_1 $ могут быть приведены к виду $ K_2 $.
\end{definition}

\begin{theorem}[О вложениях классов $ \Sigma, \Pi, \Delta $ друг в друга]
    Между классами $ \Sigma_n, \ \Pi_n $ и $ \Delta_n $ есть следующие соотношения:
    \[
        \begin{array}{ccccccccccccccc}
            \Sigma_0 &          &          &          & \Sigma_1 &          &          &          &        &          &          & \Sigma_n &          &              &        \\
            \verteq  & \searrow &          & \nearrow &          & \searrow &          & \nearrow &        &          & \nearrow &          & \searrow &              &        \\
            \Delta_0 &          & \Delta_1 &          &          &          & \Delta_2 &          & \cdots & \Delta_n &          &          &          & \Delta_{n+1} & \cdots \\
            \verteq  & \nearrow &          & \searrow &          & \nearrow &          & \searrow &        &          & \searrow &          & \nearrow &              &        \\
            \Pi_0    &          &          &          & \Pi_1    &          &          &          &        &          &          & \Pi_n    &          &              &
        \end{array}
    \]
\end{theorem}

\subsection{Нормальная форма Сколема, ее построение (на примерах).}

\begin{definition}[Нормальная форма Сколема]
    \emph{Нормальная форма Сколема} -- если это ПНФ, но только с универсальными кванторами первого порядка.
\end{definition}

\begin{note}[Сколемизация]
    Пусть формула $ \Phi $ находима в ПНФ \\$ (Q_1 x_1)(Q_2x_2)\ldots(Q_nx_n) \ \phi $, где $ \phi $ -- бескванторная формула $ Q_i \in \{\forall,\exists\} $ и пусть $ \exists x_i $ -- самый левый квантор $ \exists $ в префиксе.

    Необходимо произвести следующие операции:
    \begin{enumerate}
        \item Если левее $ \exists x_i $ ничего нет, то все вхождения $ x_i $ заменяются на новый константный символ $ c $, не принадлежащий сигнатуре. При этом константный символ $ c $ добавляется в сигнатуру.
        \item Если левее $ \exists x_i $ стоят кванторы $ \forall x_1, \forall x_2, \ldots, \forall x_k $, то все вхождения $ x_i $ в $ \phi $ заменяются на новый $ k $-местный функциональный символ $ f(x_1,x_2,\ldots,x_n) $, не принадлежащий сигнатуре. При этом функциональный символ $ f $ добавляется в сигнатуру.
        \item После выполнения указанных выше замен, выражение $ \exists x_i $ удаляется из кванторного префикса.
    \end{enumerate}
\end{note}

\subsection{Изоморфизм алгебраических систем. Теорема о сохранении значений термов и формул в изоморфных системах. Автоморфизм.}

\begin{note}
    Неформально, две системы \emph{изоморфны}, если у них все свойства совпадают.
\end{note}

\begin{example}
    $ \sigma = \{f^{(1)},c\} $
    \[
        \begin{array}{ll}
            \mathfrak{N} = <\N,\sigma>                                  & C_\mathfrak{N} = 1, \ f_\mathfrak{N}(x) = x+1         \\
            \mathfrak{B} = <\underbrace{\{b_1,b_2,\ldots\}}_{B},\sigma> & C_\mathfrak{B} = b_1, \ f_\mathfrak{B}(b_k) = b_{k+1}
        \end{array}
    \]
    \[
        \mathfrak{N}\ne\mathfrak{B}
    \]
\end{example}

\begin{definition}[Изоморфизм системы на систему]
    Пусть $ \mathfrak{N} = <A,\sigma>, \ \mathfrak{B} = <B,\sigma> $ -- алгебраические системы сигнатуры $ \sigma $. \emph{Изоморфизм системы $ \mathfrak{N} $ на систему $ \mathfrak{B} $} -- это биекция $ \alpha: A \rightarrow B $ со свойствами:
    \begin{itemize}
        \item для любого константного символа $ c \in \sigma \ \alpha(c_\mathfrak{N}) = c_\mathfrak{B} $,
        \item для любого предикатного символа $ P^{(n)}\in\sigma \ \forall a_1,\ldots,a_n \in A $
              \[
                  P_\mathfrak{N}(a_1,\ldots,a_n) = 1 \iff P_\mathfrak{B}\big(\alpha(a_1),\ldots,\alpha(a_n)\big) = 1,
              \]
        \item для любого функционального символа $ f^{(n)} \in \sigma \ \forall a_1,\ldots,a_n,b \in A $
              \[
                  b = f_\mathfrak{N}(a_1,\ldots,a_n) \iff \alpha(b) = f_\mathfrak{B}\big(\alpha(a_1),\ldots,\alpha(a_n)\big).
              \]
    \end{itemize}
\end{definition}

\begin{remark}
    Пункт 3. определения можно изложить так: $ \forall a_1,\ldots,a_n \in A $
    \[
        \alpha\big(f_\mathfrak{N}(a_1,\ldots,a_n)\big) = f_\mathfrak{B}\big(\alpha(a_1),\ldots,\alpha(a_n)\big).
    \]

    Системы $ \mathfrak{N} $ и $ \mathfrak{B} $ изоморфны ($ \mathfrak{N} \cong \mathfrak{B} $), если существует изоморфизм $ \mathfrak{N} $ на $ \mathfrak{B} $.
\end{remark}

\subsection{Элементарная теория алгебраической системы. Элементарная эквивалентность систем. Связь понятий изоморфизма и элементарной эквивалентности.}

\begin{note}
    Изоморфизм является отношением эквивалентности.

    \emph{Отношение эквивалентности}:
    \begin{itemize}
        \item рефлексивность: $ x \sim x $,
        \item симметричность: $ x \sim y \iff y \sim x $,
        \item транзитивность: $ x \sim y, \ y \sim z \implies x \sim z $.
    \end{itemize}
\end{note}

\begin{definition}[Замкнутая формула]
    Формула $ \phi $ \emph{замкнутая}, если все ее переменные связанные.
    \[
        \begin{array}{ll}
            \phi(x) \text{ -- утверждение про }x,           & \phi_1(x) \text{ студент }x \text{ -- отличник}, \\
            \phi(x) \text{ -- утверждение про всю систему}, & \phi_2 \text{ всех студентов отчислят}.
        \end{array}
    \]
\end{definition}

\begin{definition}[Элементарно эквивалентные системы]
    Системы $ \mathfrak{N} $ и $ \mathfrak{B} $ сигнатуры $ \sigma $ \emph{элементарно эквивалентны} ($ \mathfrak{N} \cong \mathfrak{B} $), если для любой замкнутой формулы $ \phi $ сигнатуры $ \sigma $
    \[
        \mathfrak{N} \vDash \phi \iff \mathfrak{B} \vDash \phi.
    \]
\end{definition}

\begin{theorem}[О связи изоморфизма и элементарной эквивалентности]
    Пусть $ \mathfrak{N} $ и $ \mathfrak{B} $ -- алгебраичесские системы сигнатуры $ \sigma $. Тогда:
    \[
        \mathfrak{N} \cong \mathfrak{B} \implies \mathfrak{N} \equiv \mathfrak{B},
    \]
    \[
        \mathfrak{N} \nequiv \mathfrak{B} \implies \mathfrak{N} \ncong \mathfrak{B}.
    \]
\end{theorem}

\subsection{Логическое следование в логике предикатов.}

\begin{definition}[Логическое следование в логике предикатов]
    Пусть $ \Gamma $ -- множество формул логики предикатов, $ \phi $ -- формула. Формула $ \phi $ \emph{логически следует} из $ \Gamma $ ($ \Gamma \vDash \phi $), если в любой алгебраической системе $ \mathfrak{N} $ и $ \forall a_1,\ldots,a_n \in \mathfrak{N} \ \big( \forall \psi \in \Gamma \ \mathfrak{N} \vDash \psi(a_1,\ldots,a_n) \big) \implies \mathfrak{N} \vDash \phi(a_1,\ldots,a_n) $.
\end{definition}

\subsection{Теория. Модель теории.}

\begin{definition}[Элементарная теория алгебраической системы]
    \emph{Элементарная теория алгебраической системы $ \mathfrak{N} $} -- это множество теорий
    \[
        \mathfrak{N} = \big\{\phi \ \big| \ \phi \text{ -- замкнутая}, \ \mathfrak{N}\vDash \phi\big\}.
    \]
\end{definition}

\begin{remark}
    \[
        \mathfrak{N} \equiv \mathfrak{B} \iff Th \mathfrak{N} = Th \mathfrak{B}.
    \]
\end{remark}

\begin{definition}[Теория]
    \emph{Теория} -- это множество замкнутых формул одной сигнатуры.
\end{definition}

\begin{definition}[Модель теории]
    \emph{Модель теории $ T $} -- это алгебраическая система $ \mathfrak{N} $, в которой истинны все формулы из $ T $ ($ \mathfrak{N} \vDash T $).
\end{definition}

\begin{note}
    $ \mathfrak{N} \vDash Th \mathfrak{B} $, то $ \mathfrak{N} \equiv \mathfrak{B} $.
\end{note}

\subsection{Исчисление предикатов (ИП) Гильберта. Свойства выводов.}

\begin{definition}[Корректная подстановка терма в формулу]
    Подставновка терма $ t $ в формулу $ \phi $ вместо переменной $ x $ \emph{корректна}, если никакая переменная терма $ t $ после подстановки не попадает в облать действия квантора по этой переменной.
\end{definition}

\begin{definition}[Исчисление предикатов Гильберта]
    Зафиксируем сигнатуру $ \sigma $. Исчисление предикатов сигнатуры $ \sigma $ состоит из следующих элементов:
    \begin{enumerate}
        \item Алфавит:
              \[
                  \mathfrak{A}_\sigma = \sigma \cup \{x_1,x_2,\ldots,\land,\lor,\lnot,\rightarrow,\leftrightarrow, (,),\forall,\exists, ,\}.
              \]
        \item Формулы исчисления предикатов (формулы логики предикатов, использующие только $ \rightarrow, \lnot $).
        \item Аксиомы ИП:
              \begin{itemize}
                  \item $A_1 \ \phi \rightarrow (\psi \rightarrow \phi) \text{ (аксиомы ИВ)} $                                                                          \\
                  \item $A_2 \ \bigl(\phi \rightarrow (\psi \rightarrow \theta)\bigr)\rightarrow \bigl((\phi \rightarrow\psi)\rightarrow(\phi \rightarrow\theta)\bigr)$ \\
                  \item $A_3 \ (\lnot\psi \rightarrow \lnot\phi) \rightarrow \bigl((\lnot \psi \rightarrow\phi)\rightarrow\psi\bigr)$                                   \\
                  \item $B_1 \ \bigl(\forall x \ \phi(x)\bigr) \rightarrow \phi(t) \text{ (аксиомы кванторов)}$                                                         \\
                  \item $B_2 \ \phi(t) \rightarrow \bigl(\exists x \ \phi(x)\bigr)$
                        (здесь $ t $ -- терм, подстановка которого в формулу $ \phi $ вместо переменной $ x $ -- корректна) \\
                  \item $\forall x \ x = x \text{ (аксиомы равенства)}$   \\
                  \item $\forall x \forall y \ (x = y \rightarrow y = x)$ \\
                  \item $\forall x \forall y \forall z \ (x = y \land y = z \rightarrow x = z)$ \\
                  \item Для предикатного символа $ P^{(n)}\in \sigma: \\ \forall x_1,\ldots,\forall x_n,\forall y_1,\ldots,\forall y_n $
                        \[
                            x_1 = y_1 \land \ldots \land x_n = y_n \land P(x_1,\ldots,x_n) \iff P(y_1,\ldots,y_n).
                        \]
                  \item Для функционального символа $ f^{(n)}\in\sigma: \\ \forall x_1,\ldots,\forall x_n,\forall y_1,\ldots,\forall y_n $
                        \[
                            x_1 = y_1 \land \ldots \land x_n = y_n \rightarrow f(x_1,\ldots,x_n) = f(y_1,\ldots,y_n).
                        \]
              \end{itemize}
        \item Правило вывода ИП:
              \[
                  \frac{\phi, \ \phi \rightarrow \psi}{\psi} \quad \text{(modus ponens)},
              \]
              \[
                  \frac{\phi \rightarrow\psi(x)}{\phi \rightarrow \big(\forall x \ \psi(x)\big)}, \ \frac{\psi(x)\rightarrow \phi}{\big(\exists x \ \psi(x)\big) \rightarrow \phi} \quad \text{(правила П. Бернайса)}
              \]
              \begin{center}
                  ($ x $ не входит свободно в $ \phi $).
              \end{center}
    \end{enumerate}
\end{definition}

\begin{statement}[Свойства выводов]
    В любой формальной системе выполнены следующие утверждения:
    \begin{enumerate}
        \item Если $\Gamma \vdash \phi $, то существует конечное подмножество $ \Gamma_0 \subset \Gamma $ такое, что $ \Gamma_0 \vdash \phi $.
        \item Если $ \Gamma \vdash \phi $ и $ \Gamma \subset \Delta $, то $ \Delta \vdash \phi $.
        \item Если для каждой формулы $ \phi \in \Delta $ выполнено $ \Gamma \vdash \phi $ и $ \Delta \vdash \psi $, то и $ \Gamma \vdash \psi $.
    \end{enumerate}
\end{statement}

\subsection{Непротиворечивая теория.}

\begin{definition}[Противоречивая, непротиворечивая теория]
    Теория $ T $ \emph{противоречивая}, если существует формула $ \phi $ такая, что одновременно $ T \vDash \phi $ и $ T \vDash \lnot\phi $. В противном случае, теория $ T $ -- \emph{непротиворечивая}.
\end{definition}

% \begin{definition}[Полная теория]
%    Непротиворечивая теория $ T $ \emph{полна} (в данной сигнатуре), если для любой замкнутой формулы $ \phi $ этой сигнатуры либо формула $ \phi $, либо $ \lnot\phi $ выводится из $ T $.
% \end{definition}

\subsection{О существовании модели (без доказательства).}

\begin{theorem}[Теорема о существовании модели]
    Каждая непротиворечивая теория имеет модель.
\end{theorem}

\subsection{Теорема о связи выводимости и противоречивости.}

\begin{theorem}[О связи выводимости и противоречивости]
    Пусть $ T $ -- теория, $ \phi $ -- замкнутая формула.

    Тогда $ T \vdash \phi \iff $ теория $ T \cup \{\lnot\phi\} $ противоречива.
\end{theorem}

\subsection{Теоремы о корректности и полноте ИП.}

\begin{theorem}[Корректность ИП]
    $ T $ -- множетсво замкнутых формул сигнатуры $ \sigma $, $ \phi $ -- формула сигнатуры $ \sigma $. Тогда:
    \[
        T \vdash \phi \implies T \vDash \phi.
    \]
\end{theorem}

\begin{theorem}[Полнота ИП]
    $ T $ -- множетсво замкнутых формул сигнатуры $ \sigma $, $ \phi $ -- формула сигнатуры $ \sigma $. Тогда:
    \[
        T \vdash \phi \Leftarrow T \vDash \phi.
    \]
\end{theorem}

\subsection{Теорема компактности.}

\begin{theorem}[Теорема компактности]
    Теория имеет модель тогда и только тогда, когда ее конечная подтеория имеет модель.
\end{theorem}

\subsection{Обоснование нестандартного анализа (построение алгебраической системы, элементарно эквивалентной полю вещественных чисел, содержащей бесконечно малые элементы).}

\begin{theorem}[А. Робинсон]
    Существует алгебраическая система $ \R^* $ с бесконечно малыми элементами, эквивалентная системе
    \[
        R = <\R,+,\cdot,-,/,0,1,<> \text{ -- гипервещественные числа}.
    \]
\end{theorem}

\subsection{Метод резолюций для логики предикатов (без доказательства корректности).}

\begin{center}
    {\Huge НУЖНО НАЙТИ}
\end{center}
